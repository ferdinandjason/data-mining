
% Default to the notebook output style

    


% Inherit from the specified cell style.




    
\documentclass{report}

    
    
    \usepackage[T1]{fontenc}
    % Nicer default font (+ math font) than Computer Modern for most use cases
    \usepackage{mathpazo}

    % Basic figure setup, for now with no caption control since it's done
    % automatically by Pandoc (which extracts ![](path) syntax from Markdown).
    \usepackage{graphicx}
    % We will generate all images so they have a width \maxwidth. This means
    % that they will get their normal width if they fit onto the page, but
    % are scaled down if they would overflow the margins.
    \makeatletter
    \def\maxwidth{\ifdim\Gin@nat@width>\linewidth\linewidth
    \else\Gin@nat@width\fi}
    \makeatother
    \let\Oldincludegraphics\includegraphics
    % Set max figure width to be 80% of text width, for now hardcoded.
    \renewcommand{\includegraphics}[1]{\Oldincludegraphics[width=.8\maxwidth]{#1}}
    % Ensure that by default, figures have no caption (until we provide a
    % proper Figure object with a Caption API and a way to capture that
    % in the conversion process - todo).
    \usepackage{caption}
    \DeclareCaptionLabelFormat{nolabel}{}
    \captionsetup{labelformat=nolabel}

    \usepackage{adjustbox} % Used to constrain images to a maximum size 
    \usepackage{xcolor} % Allow colors to be defined
    \usepackage{enumerate} % Needed for markdown enumerations to work
    \usepackage{geometry} % Used to adjust the document margins
    \usepackage{amsmath} % Equations
    \usepackage{amssymb} % Equations
    \usepackage{textcomp} % defines textquotesingle
    % Hack from http://tex.stackexchange.com/a/47451/13684:
    \AtBeginDocument{%
        \def\PYZsq{\textquotesingle}% Upright quotes in Pygmentized code
    }
    \usepackage{upquote} % Upright quotes for verbatim code
    \usepackage{eurosym} % defines \euro
    \usepackage[mathletters]{ucs} % Extended unicode (utf-8) support
    \usepackage[utf8x]{inputenc} % Allow utf-8 characters in the tex document
    \usepackage{fancyvrb} % verbatim replacement that allows latex
    \usepackage{grffile} % extends the file name processing of package graphics 
                         % to support a larger range 
    % The hyperref package gives us a pdf with properly built
    % internal navigation ('pdf bookmarks' for the table of contents,
    % internal cross-reference links, web links for URLs, etc.)
    \usepackage{hyperref}
    \usepackage{longtable} % longtable support required by pandoc >1.10
    \usepackage{booktabs}  % table support for pandoc > 1.12.2
    \usepackage[inline]{enumitem} % IRkernel/repr support (it uses the enumerate* environment)
    \usepackage[normalem]{ulem} % ulem is needed to support strikethroughs (\sout)
                                % normalem makes italics be italics, not underlines
    

    
    
    % Colors for the hyperref package
    \definecolor{urlcolor}{rgb}{0,.145,.698}
    \definecolor{linkcolor}{rgb}{.71,0.21,0.01}
    \definecolor{citecolor}{rgb}{.12,.54,.11}

    % ANSI colors
    \definecolor{ansi-black}{HTML}{3E424D}
    \definecolor{ansi-black-intense}{HTML}{282C36}
    \definecolor{ansi-red}{HTML}{E75C58}
    \definecolor{ansi-red-intense}{HTML}{B22B31}
    \definecolor{ansi-green}{HTML}{00A250}
    \definecolor{ansi-green-intense}{HTML}{007427}
    \definecolor{ansi-yellow}{HTML}{DDB62B}
    \definecolor{ansi-yellow-intense}{HTML}{B27D12}
    \definecolor{ansi-blue}{HTML}{208FFB}
    \definecolor{ansi-blue-intense}{HTML}{0065CA}
    \definecolor{ansi-magenta}{HTML}{D160C4}
    \definecolor{ansi-magenta-intense}{HTML}{A03196}
    \definecolor{ansi-cyan}{HTML}{60C6C8}
    \definecolor{ansi-cyan-intense}{HTML}{258F8F}
    \definecolor{ansi-white}{HTML}{C5C1B4}
    \definecolor{ansi-white-intense}{HTML}{A1A6B2}

    % commands and environments needed by pandoc snippets
    % extracted from the output of `pandoc -s`
    \providecommand{\tightlist}{%
      \setlength{\itemsep}{0pt}\setlength{\parskip}{0pt}}
    \DefineVerbatimEnvironment{Highlighting}{Verbatim}{commandchars=\\\{\}}
    % Add ',fontsize=\small' for more characters per line
    \newenvironment{Shaded}{}{}
    \newcommand{\KeywordTok}[1]{\textcolor[rgb]{0.00,0.44,0.13}{\textbf{{#1}}}}
    \newcommand{\DataTypeTok}[1]{\textcolor[rgb]{0.56,0.13,0.00}{{#1}}}
    \newcommand{\DecValTok}[1]{\textcolor[rgb]{0.25,0.63,0.44}{{#1}}}
    \newcommand{\BaseNTok}[1]{\textcolor[rgb]{0.25,0.63,0.44}{{#1}}}
    \newcommand{\FloatTok}[1]{\textcolor[rgb]{0.25,0.63,0.44}{{#1}}}
    \newcommand{\CharTok}[1]{\textcolor[rgb]{0.25,0.44,0.63}{{#1}}}
    \newcommand{\StringTok}[1]{\textcolor[rgb]{0.25,0.44,0.63}{{#1}}}
    \newcommand{\CommentTok}[1]{\textcolor[rgb]{0.38,0.63,0.69}{\textit{{#1}}}}
    \newcommand{\OtherTok}[1]{\textcolor[rgb]{0.00,0.44,0.13}{{#1}}}
    \newcommand{\AlertTok}[1]{\textcolor[rgb]{1.00,0.00,0.00}{\textbf{{#1}}}}
    \newcommand{\FunctionTok}[1]{\textcolor[rgb]{0.02,0.16,0.49}{{#1}}}
    \newcommand{\RegionMarkerTok}[1]{{#1}}
    \newcommand{\ErrorTok}[1]{\textcolor[rgb]{1.00,0.00,0.00}{\textbf{{#1}}}}
    \newcommand{\NormalTok}[1]{{#1}}
    
    % Additional commands for more recent versions of Pandoc
    \newcommand{\ConstantTok}[1]{\textcolor[rgb]{0.53,0.00,0.00}{{#1}}}
    \newcommand{\SpecialCharTok}[1]{\textcolor[rgb]{0.25,0.44,0.63}{{#1}}}
    \newcommand{\VerbatimStringTok}[1]{\textcolor[rgb]{0.25,0.44,0.63}{{#1}}}
    \newcommand{\SpecialStringTok}[1]{\textcolor[rgb]{0.73,0.40,0.53}{{#1}}}
    \newcommand{\ImportTok}[1]{{#1}}
    \newcommand{\DocumentationTok}[1]{\textcolor[rgb]{0.73,0.13,0.13}{\textit{{#1}}}}
    \newcommand{\AnnotationTok}[1]{\textcolor[rgb]{0.38,0.63,0.69}{\textbf{\textit{{#1}}}}}
    \newcommand{\CommentVarTok}[1]{\textcolor[rgb]{0.38,0.63,0.69}{\textbf{\textit{{#1}}}}}
    \newcommand{\VariableTok}[1]{\textcolor[rgb]{0.10,0.09,0.49}{{#1}}}
    \newcommand{\ControlFlowTok}[1]{\textcolor[rgb]{0.00,0.44,0.13}{\textbf{{#1}}}}
    \newcommand{\OperatorTok}[1]{\textcolor[rgb]{0.40,0.40,0.40}{{#1}}}
    \newcommand{\BuiltInTok}[1]{{#1}}
    \newcommand{\ExtensionTok}[1]{{#1}}
    \newcommand{\PreprocessorTok}[1]{\textcolor[rgb]{0.74,0.48,0.00}{{#1}}}
    \newcommand{\AttributeTok}[1]{\textcolor[rgb]{0.49,0.56,0.16}{{#1}}}
    \newcommand{\InformationTok}[1]{\textcolor[rgb]{0.38,0.63,0.69}{\textbf{\textit{{#1}}}}}
    \newcommand{\WarningTok}[1]{\textcolor[rgb]{0.38,0.63,0.69}{\textbf{\textit{{#1}}}}}
    
    
    % Define a nice break command that doesn't care if a line doesn't already
    % exist.
    \def\br{\hspace*{\fill} \\* }
    % Math Jax compatability definitions
    \def\gt{>}
    \def\lt{<}
    % Document parameters
    \title{tugas-1}
    
    
    

    % Pygments definitions
    
\makeatletter
\def\PY@reset{\let\PY@it=\relax \let\PY@bf=\relax%
    \let\PY@ul=\relax \let\PY@tc=\relax%
    \let\PY@bc=\relax \let\PY@ff=\relax}
\def\PY@tok#1{\csname PY@tok@#1\endcsname}
\def\PY@toks#1+{\ifx\relax#1\empty\else%
    \PY@tok{#1}\expandafter\PY@toks\fi}
\def\PY@do#1{\PY@bc{\PY@tc{\PY@ul{%
    \PY@it{\PY@bf{\PY@ff{#1}}}}}}}
\def\PY#1#2{\PY@reset\PY@toks#1+\relax+\PY@do{#2}}

\expandafter\def\csname PY@tok@w\endcsname{\def\PY@tc##1{\textcolor[rgb]{0.73,0.73,0.73}{##1}}}
\expandafter\def\csname PY@tok@c\endcsname{\let\PY@it=\textit\def\PY@tc##1{\textcolor[rgb]{0.25,0.50,0.50}{##1}}}
\expandafter\def\csname PY@tok@cp\endcsname{\def\PY@tc##1{\textcolor[rgb]{0.74,0.48,0.00}{##1}}}
\expandafter\def\csname PY@tok@k\endcsname{\let\PY@bf=\textbf\def\PY@tc##1{\textcolor[rgb]{0.00,0.50,0.00}{##1}}}
\expandafter\def\csname PY@tok@kp\endcsname{\def\PY@tc##1{\textcolor[rgb]{0.00,0.50,0.00}{##1}}}
\expandafter\def\csname PY@tok@kt\endcsname{\def\PY@tc##1{\textcolor[rgb]{0.69,0.00,0.25}{##1}}}
\expandafter\def\csname PY@tok@o\endcsname{\def\PY@tc##1{\textcolor[rgb]{0.40,0.40,0.40}{##1}}}
\expandafter\def\csname PY@tok@ow\endcsname{\let\PY@bf=\textbf\def\PY@tc##1{\textcolor[rgb]{0.67,0.13,1.00}{##1}}}
\expandafter\def\csname PY@tok@nb\endcsname{\def\PY@tc##1{\textcolor[rgb]{0.00,0.50,0.00}{##1}}}
\expandafter\def\csname PY@tok@nf\endcsname{\def\PY@tc##1{\textcolor[rgb]{0.00,0.00,1.00}{##1}}}
\expandafter\def\csname PY@tok@nc\endcsname{\let\PY@bf=\textbf\def\PY@tc##1{\textcolor[rgb]{0.00,0.00,1.00}{##1}}}
\expandafter\def\csname PY@tok@nn\endcsname{\let\PY@bf=\textbf\def\PY@tc##1{\textcolor[rgb]{0.00,0.00,1.00}{##1}}}
\expandafter\def\csname PY@tok@ne\endcsname{\let\PY@bf=\textbf\def\PY@tc##1{\textcolor[rgb]{0.82,0.25,0.23}{##1}}}
\expandafter\def\csname PY@tok@nv\endcsname{\def\PY@tc##1{\textcolor[rgb]{0.10,0.09,0.49}{##1}}}
\expandafter\def\csname PY@tok@no\endcsname{\def\PY@tc##1{\textcolor[rgb]{0.53,0.00,0.00}{##1}}}
\expandafter\def\csname PY@tok@nl\endcsname{\def\PY@tc##1{\textcolor[rgb]{0.63,0.63,0.00}{##1}}}
\expandafter\def\csname PY@tok@ni\endcsname{\let\PY@bf=\textbf\def\PY@tc##1{\textcolor[rgb]{0.60,0.60,0.60}{##1}}}
\expandafter\def\csname PY@tok@na\endcsname{\def\PY@tc##1{\textcolor[rgb]{0.49,0.56,0.16}{##1}}}
\expandafter\def\csname PY@tok@nt\endcsname{\let\PY@bf=\textbf\def\PY@tc##1{\textcolor[rgb]{0.00,0.50,0.00}{##1}}}
\expandafter\def\csname PY@tok@nd\endcsname{\def\PY@tc##1{\textcolor[rgb]{0.67,0.13,1.00}{##1}}}
\expandafter\def\csname PY@tok@s\endcsname{\def\PY@tc##1{\textcolor[rgb]{0.73,0.13,0.13}{##1}}}
\expandafter\def\csname PY@tok@sd\endcsname{\let\PY@it=\textit\def\PY@tc##1{\textcolor[rgb]{0.73,0.13,0.13}{##1}}}
\expandafter\def\csname PY@tok@si\endcsname{\let\PY@bf=\textbf\def\PY@tc##1{\textcolor[rgb]{0.73,0.40,0.53}{##1}}}
\expandafter\def\csname PY@tok@se\endcsname{\let\PY@bf=\textbf\def\PY@tc##1{\textcolor[rgb]{0.73,0.40,0.13}{##1}}}
\expandafter\def\csname PY@tok@sr\endcsname{\def\PY@tc##1{\textcolor[rgb]{0.73,0.40,0.53}{##1}}}
\expandafter\def\csname PY@tok@ss\endcsname{\def\PY@tc##1{\textcolor[rgb]{0.10,0.09,0.49}{##1}}}
\expandafter\def\csname PY@tok@sx\endcsname{\def\PY@tc##1{\textcolor[rgb]{0.00,0.50,0.00}{##1}}}
\expandafter\def\csname PY@tok@m\endcsname{\def\PY@tc##1{\textcolor[rgb]{0.40,0.40,0.40}{##1}}}
\expandafter\def\csname PY@tok@gh\endcsname{\let\PY@bf=\textbf\def\PY@tc##1{\textcolor[rgb]{0.00,0.00,0.50}{##1}}}
\expandafter\def\csname PY@tok@gu\endcsname{\let\PY@bf=\textbf\def\PY@tc##1{\textcolor[rgb]{0.50,0.00,0.50}{##1}}}
\expandafter\def\csname PY@tok@gd\endcsname{\def\PY@tc##1{\textcolor[rgb]{0.63,0.00,0.00}{##1}}}
\expandafter\def\csname PY@tok@gi\endcsname{\def\PY@tc##1{\textcolor[rgb]{0.00,0.63,0.00}{##1}}}
\expandafter\def\csname PY@tok@gr\endcsname{\def\PY@tc##1{\textcolor[rgb]{1.00,0.00,0.00}{##1}}}
\expandafter\def\csname PY@tok@ge\endcsname{\let\PY@it=\textit}
\expandafter\def\csname PY@tok@gs\endcsname{\let\PY@bf=\textbf}
\expandafter\def\csname PY@tok@gp\endcsname{\let\PY@bf=\textbf\def\PY@tc##1{\textcolor[rgb]{0.00,0.00,0.50}{##1}}}
\expandafter\def\csname PY@tok@go\endcsname{\def\PY@tc##1{\textcolor[rgb]{0.53,0.53,0.53}{##1}}}
\expandafter\def\csname PY@tok@gt\endcsname{\def\PY@tc##1{\textcolor[rgb]{0.00,0.27,0.87}{##1}}}
\expandafter\def\csname PY@tok@err\endcsname{\def\PY@bc##1{\setlength{\fboxsep}{0pt}\fcolorbox[rgb]{1.00,0.00,0.00}{1,1,1}{\strut ##1}}}
\expandafter\def\csname PY@tok@kc\endcsname{\let\PY@bf=\textbf\def\PY@tc##1{\textcolor[rgb]{0.00,0.50,0.00}{##1}}}
\expandafter\def\csname PY@tok@kd\endcsname{\let\PY@bf=\textbf\def\PY@tc##1{\textcolor[rgb]{0.00,0.50,0.00}{##1}}}
\expandafter\def\csname PY@tok@kn\endcsname{\let\PY@bf=\textbf\def\PY@tc##1{\textcolor[rgb]{0.00,0.50,0.00}{##1}}}
\expandafter\def\csname PY@tok@kr\endcsname{\let\PY@bf=\textbf\def\PY@tc##1{\textcolor[rgb]{0.00,0.50,0.00}{##1}}}
\expandafter\def\csname PY@tok@bp\endcsname{\def\PY@tc##1{\textcolor[rgb]{0.00,0.50,0.00}{##1}}}
\expandafter\def\csname PY@tok@fm\endcsname{\def\PY@tc##1{\textcolor[rgb]{0.00,0.00,1.00}{##1}}}
\expandafter\def\csname PY@tok@vc\endcsname{\def\PY@tc##1{\textcolor[rgb]{0.10,0.09,0.49}{##1}}}
\expandafter\def\csname PY@tok@vg\endcsname{\def\PY@tc##1{\textcolor[rgb]{0.10,0.09,0.49}{##1}}}
\expandafter\def\csname PY@tok@vi\endcsname{\def\PY@tc##1{\textcolor[rgb]{0.10,0.09,0.49}{##1}}}
\expandafter\def\csname PY@tok@vm\endcsname{\def\PY@tc##1{\textcolor[rgb]{0.10,0.09,0.49}{##1}}}
\expandafter\def\csname PY@tok@sa\endcsname{\def\PY@tc##1{\textcolor[rgb]{0.73,0.13,0.13}{##1}}}
\expandafter\def\csname PY@tok@sb\endcsname{\def\PY@tc##1{\textcolor[rgb]{0.73,0.13,0.13}{##1}}}
\expandafter\def\csname PY@tok@sc\endcsname{\def\PY@tc##1{\textcolor[rgb]{0.73,0.13,0.13}{##1}}}
\expandafter\def\csname PY@tok@dl\endcsname{\def\PY@tc##1{\textcolor[rgb]{0.73,0.13,0.13}{##1}}}
\expandafter\def\csname PY@tok@s2\endcsname{\def\PY@tc##1{\textcolor[rgb]{0.73,0.13,0.13}{##1}}}
\expandafter\def\csname PY@tok@sh\endcsname{\def\PY@tc##1{\textcolor[rgb]{0.73,0.13,0.13}{##1}}}
\expandafter\def\csname PY@tok@s1\endcsname{\def\PY@tc##1{\textcolor[rgb]{0.73,0.13,0.13}{##1}}}
\expandafter\def\csname PY@tok@mb\endcsname{\def\PY@tc##1{\textcolor[rgb]{0.40,0.40,0.40}{##1}}}
\expandafter\def\csname PY@tok@mf\endcsname{\def\PY@tc##1{\textcolor[rgb]{0.40,0.40,0.40}{##1}}}
\expandafter\def\csname PY@tok@mh\endcsname{\def\PY@tc##1{\textcolor[rgb]{0.40,0.40,0.40}{##1}}}
\expandafter\def\csname PY@tok@mi\endcsname{\def\PY@tc##1{\textcolor[rgb]{0.40,0.40,0.40}{##1}}}
\expandafter\def\csname PY@tok@il\endcsname{\def\PY@tc##1{\textcolor[rgb]{0.40,0.40,0.40}{##1}}}
\expandafter\def\csname PY@tok@mo\endcsname{\def\PY@tc##1{\textcolor[rgb]{0.40,0.40,0.40}{##1}}}
\expandafter\def\csname PY@tok@ch\endcsname{\let\PY@it=\textit\def\PY@tc##1{\textcolor[rgb]{0.25,0.50,0.50}{##1}}}
\expandafter\def\csname PY@tok@cm\endcsname{\let\PY@it=\textit\def\PY@tc##1{\textcolor[rgb]{0.25,0.50,0.50}{##1}}}
\expandafter\def\csname PY@tok@cpf\endcsname{\let\PY@it=\textit\def\PY@tc##1{\textcolor[rgb]{0.25,0.50,0.50}{##1}}}
\expandafter\def\csname PY@tok@c1\endcsname{\let\PY@it=\textit\def\PY@tc##1{\textcolor[rgb]{0.25,0.50,0.50}{##1}}}
\expandafter\def\csname PY@tok@cs\endcsname{\let\PY@it=\textit\def\PY@tc##1{\textcolor[rgb]{0.25,0.50,0.50}{##1}}}

\def\PYZbs{\char`\\}
\def\PYZus{\char`\_}
\def\PYZob{\char`\{}
\def\PYZcb{\char`\}}
\def\PYZca{\char`\^}
\def\PYZam{\char`\&}
\def\PYZlt{\char`\<}
\def\PYZgt{\char`\>}
\def\PYZsh{\char`\#}
\def\PYZpc{\char`\%}
\def\PYZdl{\char`\$}
\def\PYZhy{\char`\-}
\def\PYZsq{\char`\'}
\def\PYZdq{\char`\"}
\def\PYZti{\char`\~}
% for compatibility with earlier versions
\def\PYZat{@}
\def\PYZlb{[}
\def\PYZrb{]}
\makeatother


    % Exact colors from NB
    \definecolor{incolor}{rgb}{0.0, 0.0, 0.5}
    \definecolor{outcolor}{rgb}{0.545, 0.0, 0.0}



    
    % Prevent overflowing lines due to hard-to-break entities
    \sloppy 
    % Setup hyperref package
    \hypersetup{
      breaklinks=true,  % so long urls are correctly broken across lines
      colorlinks=true,
      urlcolor=urlcolor,
      linkcolor=linkcolor,
      citecolor=citecolor,
      }
    % Slightly bigger margins than the latex defaults
    
    \geometry{verbose,tmargin=1in,bmargin=1in,lmargin=1in,rmargin=1in}
    
    

    \begin{document}
    
    
    
    \maketitle
    
    
    \tableofcontents


    
\chapter{Adult Data Set Analysis}\label{adult-data-set-analysis}

This dataset analysis task is carried out by: - Nuzha Musyafira -
05111640000014 - Ferdinand Jason Gondowijoyo - 05111640000033 - Nurlita
Dhuha Fatmawati - 05111640000092 - Jonathan Rehuel Lewerissa -
05111640000105

\section{Introduction}\label{introduction}

On this dataset analysis task, we will analyze the Adult Data Set. The
Adult Data Set (also known as the Census Dataset) is a dataset that aims
to predict whether a person's income exceeds \$50000 per year based on
their census data.

This data set can be downloaded from
\textbf{\url{https://archive.ics.uci.edu/ml/datasets/adult}}.

\section{Preparation}\label{preparation}

Let's first \texttt{import} some libraries that we are going to need for
our analysis.

    \begin{Verbatim}[commandchars=\\\{\}]
{\color{incolor}In [{\color{incolor}1}]:} \PY{k+kn}{import} \PY{n+nn}{math}
        
        \PY{k+kn}{import} \PY{n+nn}{pandas} \PY{k}{as} \PY{n+nn}{pd}
        \PY{k+kn}{import} \PY{n+nn}{numpy} \PY{k}{as} \PY{n+nn}{np}
        \PY{k+kn}{import} \PY{n+nn}{seaborn} \PY{k}{as} \PY{n+nn}{sns}
        \PY{k+kn}{import} \PY{n+nn}{matplotlib}\PY{n+nn}{.}\PY{n+nn}{pyplot} \PY{k}{as} \PY{n+nn}{plt}
        
        \PY{k+kn}{import} \PY{n+nn}{sklearn}\PY{n+nn}{.}\PY{n+nn}{preprocessing} \PY{k}{as} \PY{n+nn}{preprocessing}
        
        \PY{k+kn}{from} \PY{n+nn}{sklearn}\PY{n+nn}{.}\PY{n+nn}{impute} \PY{k}{import} \PY{n}{SimpleImputer}
        
        \PY{o}{\PYZpc{}}\PY{k}{matplotlib} inline
\end{Verbatim}


Then, we need to read the adult dataset from \texttt{data/adult.csv}
which contains comma separated columns and mark the values \texttt{?} as
missing data points

    \begin{Verbatim}[commandchars=\\\{\}]
{\color{incolor}In [{\color{incolor}2}]:} \PY{n}{original\PYZus{}data} \PY{o}{=} \PY{n}{pd}\PY{o}{.}\PY{n}{read\PYZus{}csv}\PY{p}{(}
            \PY{l+s+s2}{\PYZdq{}}\PY{l+s+s2}{data/adult.csv}\PY{l+s+s2}{\PYZdq{}}\PY{p}{,}
            \PY{n}{names}\PY{o}{=}\PY{p}{[}\PY{l+s+s2}{\PYZdq{}}\PY{l+s+s2}{Age}\PY{l+s+s2}{\PYZdq{}}\PY{p}{,} \PY{l+s+s2}{\PYZdq{}}\PY{l+s+s2}{Workclass}\PY{l+s+s2}{\PYZdq{}}\PY{p}{,} \PY{l+s+s2}{\PYZdq{}}\PY{l+s+s2}{fnlwgt}\PY{l+s+s2}{\PYZdq{}}\PY{p}{,} \PY{l+s+s2}{\PYZdq{}}\PY{l+s+s2}{Education}\PY{l+s+s2}{\PYZdq{}}\PY{p}{,} \PY{l+s+s2}{\PYZdq{}}\PY{l+s+s2}{Education\PYZhy{}Num}\PY{l+s+s2}{\PYZdq{}}\PY{p}{,} \PY{l+s+s2}{\PYZdq{}}\PY{l+s+s2}{Martial Status}\PY{l+s+s2}{\PYZdq{}}\PY{p}{,}
                \PY{l+s+s2}{\PYZdq{}}\PY{l+s+s2}{Occupation}\PY{l+s+s2}{\PYZdq{}}\PY{p}{,} \PY{l+s+s2}{\PYZdq{}}\PY{l+s+s2}{Relationship}\PY{l+s+s2}{\PYZdq{}}\PY{p}{,} \PY{l+s+s2}{\PYZdq{}}\PY{l+s+s2}{Race}\PY{l+s+s2}{\PYZdq{}}\PY{p}{,} \PY{l+s+s2}{\PYZdq{}}\PY{l+s+s2}{Sex}\PY{l+s+s2}{\PYZdq{}}\PY{p}{,} \PY{l+s+s2}{\PYZdq{}}\PY{l+s+s2}{Capital Gain}\PY{l+s+s2}{\PYZdq{}}\PY{p}{,} \PY{l+s+s2}{\PYZdq{}}\PY{l+s+s2}{Capital Loss}\PY{l+s+s2}{\PYZdq{}}\PY{p}{,}
                \PY{l+s+s2}{\PYZdq{}}\PY{l+s+s2}{Hours per week}\PY{l+s+s2}{\PYZdq{}}\PY{p}{,} \PY{l+s+s2}{\PYZdq{}}\PY{l+s+s2}{Country}\PY{l+s+s2}{\PYZdq{}}\PY{p}{,} \PY{l+s+s2}{\PYZdq{}}\PY{l+s+s2}{Target}\PY{l+s+s2}{\PYZdq{}}\PY{p}{]}\PY{p}{,}
            \PY{n}{sep}\PY{o}{=}\PY{l+s+sa}{r}\PY{l+s+s1}{\PYZsq{}}\PY{l+s+s1}{\PYZbs{}}\PY{l+s+s1}{s*,}\PY{l+s+s1}{\PYZbs{}}\PY{l+s+s1}{s*}\PY{l+s+s1}{\PYZsq{}}\PY{p}{,}
            \PY{n}{engine}\PY{o}{=}\PY{l+s+s1}{\PYZsq{}}\PY{l+s+s1}{python}\PY{l+s+s1}{\PYZsq{}}\PY{p}{,}
            \PY{n}{na\PYZus{}values}\PY{o}{=}\PY{l+s+s2}{\PYZdq{}}\PY{l+s+s2}{?}\PY{l+s+s2}{\PYZdq{}}\PY{p}{)}
        
        \PY{n}{original\PYZus{}data}\PY{o}{.}\PY{n}{head}\PY{p}{(}\PY{p}{)}
\end{Verbatim}


\begin{Verbatim}[commandchars=\\\{\}]
{\color{outcolor}Out[{\color{outcolor}2}]:}    Age         Workclass  fnlwgt  Education  Education-Num  \textbackslash{}
        0   39         State-gov   77516  Bachelors             13   
        1   50  Self-emp-not-inc   83311  Bachelors             13   
        2   38           Private  215646    HS-grad              9   
        3   53           Private  234721       11th              7   
        4   28           Private  338409  Bachelors             13   
        
               Martial Status         Occupation   Relationship   Race     Sex  \textbackslash{}
        0       Never-married       Adm-clerical  Not-in-family  White    Male   
        1  Married-civ-spouse    Exec-managerial        Husband  White    Male   
        2            Divorced  Handlers-cleaners  Not-in-family  White    Male   
        3  Married-civ-spouse  Handlers-cleaners        Husband  Black    Male   
        4  Married-civ-spouse     Prof-specialty           Wife  Black  Female   
        
           Capital Gain  Capital Loss  Hours per week        Country Target  
        0          2174             0              40  United-States  <=50K  
        1             0             0              13  United-States  <=50K  
        2             0             0              40  United-States  <=50K  
        3             0             0              40  United-States  <=50K  
        4             0             0              40           Cuba  <=50K  
\end{Verbatim}
            
\section{Data Insight}\label{data-insight}

First, we need to see the general statistical information of the
dataset.

    \begin{Verbatim}[commandchars=\\\{\}]
{\color{incolor}In [{\color{incolor}3}]:} \PY{k}{def} \PY{n+nf}{summarize\PYZus{}data}\PY{p}{(}\PY{n}{df}\PY{p}{)}\PY{p}{:}
            \PY{n+nb}{print}\PY{p}{(}\PY{l+s+s1}{\PYZsq{}}\PY{l+s+s1}{Continuous Data : }\PY{l+s+s1}{\PYZsq{}}\PY{p}{)}
            \PY{n+nb}{print}\PY{p}{(}\PY{n}{df}\PY{o}{.}\PY{n}{describe}\PY{p}{(}\PY{p}{)}\PY{p}{)}
            \PY{n+nb}{print}\PY{p}{(}\PY{l+s+s1}{\PYZsq{}}\PY{l+s+se}{\PYZbs{}n}\PY{l+s+se}{\PYZbs{}n}\PY{l+s+s1}{\PYZsq{}}\PY{p}{)}
            \PY{n+nb}{print}\PY{p}{(}\PY{l+s+s1}{\PYZsq{}}\PY{l+s+s1}{Categorical Data : }\PY{l+s+s1}{\PYZsq{}}\PY{p}{)}
            \PY{k}{for} \PY{n}{column} \PY{o+ow}{in} \PY{n}{df}\PY{o}{.}\PY{n}{columns}\PY{p}{:}
                \PY{k}{if} \PY{n}{df}\PY{o}{.}\PY{n}{dtypes}\PY{p}{[}\PY{n}{column}\PY{p}{]} \PY{o}{==} \PY{n}{np}\PY{o}{.}\PY{n}{object} \PY{p}{:} \PY{c+c1}{\PYZsh{} Categorical Data}
                    \PY{n+nb}{print}\PY{p}{(}\PY{n}{column}\PY{p}{)}
                    \PY{n+nb}{print}\PY{p}{(}\PY{n}{df}\PY{p}{[}\PY{n}{column}\PY{p}{]}\PY{o}{.}\PY{n}{value\PYZus{}counts}\PY{p}{(}\PY{p}{)}\PY{p}{)}
                \PY{n+nb}{print}\PY{p}{(}\PY{p}{)}
        
        \PY{n}{summarize\PYZus{}data}\PY{p}{(}\PY{n}{original\PYZus{}data}\PY{p}{)}
\end{Verbatim}


    \begin{Verbatim}[commandchars=\\\{\}]
Continuous Data : 
                Age        fnlwgt  Education-Num  Capital Gain  Capital Loss  \textbackslash{}
count  32561.000000  3.256100e+04   32561.000000  32561.000000  32561.000000   
mean      38.581647  1.897784e+05      10.080679   1077.648844     87.303830   
std       13.640433  1.055500e+05       2.572720   7385.292085    402.960219   
min       17.000000  1.228500e+04       1.000000      0.000000      0.000000   
25\%       28.000000  1.178270e+05       9.000000      0.000000      0.000000   
50\%       37.000000  1.783560e+05      10.000000      0.000000      0.000000   
75\%       48.000000  2.370510e+05      12.000000      0.000000      0.000000   
max       90.000000  1.484705e+06      16.000000  99999.000000   4356.000000   

       Hours per week  
count    32561.000000  
mean        40.437456  
std         12.347429  
min          1.000000  
25\%         40.000000  
50\%         40.000000  
75\%         45.000000  
max         99.000000  



Categorical Data : 

Workclass
Private             22696
Self-emp-not-inc     2541
Local-gov            2093
State-gov            1298
Self-emp-inc         1116
Federal-gov           960
Without-pay            14
Never-worked            7
Name: Workclass, dtype: int64


Education
HS-grad         10501
Some-college     7291
Bachelors        5355
Masters          1723
Assoc-voc        1382
11th             1175
Assoc-acdm       1067
10th              933
7th-8th           646
Prof-school       576
9th               514
12th              433
Doctorate         413
5th-6th           333
1st-4th           168
Preschool          51
Name: Education, dtype: int64


Martial Status
Married-civ-spouse       14976
Never-married            10683
Divorced                  4443
Separated                 1025
Widowed                    993
Married-spouse-absent      418
Married-AF-spouse           23
Name: Martial Status, dtype: int64

Occupation
Prof-specialty       4140
Craft-repair         4099
Exec-managerial      4066
Adm-clerical         3770
Sales                3650
Other-service        3295
Machine-op-inspct    2002
Transport-moving     1597
Handlers-cleaners    1370
Farming-fishing       994
Tech-support          928
Protective-serv       649
Priv-house-serv       149
Armed-Forces            9
Name: Occupation, dtype: int64

Relationship
Husband           13193
Not-in-family      8305
Own-child          5068
Unmarried          3446
Wife               1568
Other-relative      981
Name: Relationship, dtype: int64

Race
White                 27816
Black                  3124
Asian-Pac-Islander     1039
Amer-Indian-Eskimo      311
Other                   271
Name: Race, dtype: int64

Sex
Male      21790
Female    10771
Name: Sex, dtype: int64




Country
United-States                 29170
Mexico                          643
Philippines                     198
Germany                         137
Canada                          121
Puerto-Rico                     114
El-Salvador                     106
India                           100
Cuba                             95
England                          90
Jamaica                          81
South                            80
China                            75
Italy                            73
Dominican-Republic               70
Vietnam                          67
Guatemala                        64
Japan                            62
Poland                           60
Columbia                         59
Taiwan                           51
Haiti                            44
Iran                             43
Portugal                         37
Nicaragua                        34
Peru                             31
France                           29
Greece                           29
Ecuador                          28
Ireland                          24
Hong                             20
Trinadad\&Tobago                  19
Cambodia                         19
Thailand                         18
Laos                             18
Yugoslavia                       16
Outlying-US(Guam-USVI-etc)       14
Honduras                         13
Hungary                          13
Scotland                         12
Holand-Netherlands                1
Name: Country, dtype: int64

Target
<=50K    24720
>50K      7841
Name: Target, dtype: int64


    \end{Verbatim}

\subsection{Data Dictionary}\label{data-dictionary}

\begin{enumerate}
\def\labelenumi{\arabic{enumi}.}
\tightlist
\item
  Categorial Attributes

  \begin{itemize}
  \tightlist
  \item
    workclass: (categorical) Private, Self-emp-not-inc, Self-emp-inc,
    Federal-gov, Local-gov, State-gov, Without-pay, Never-worked.

    \begin{itemize}
    \tightlist
    \item
      Individual work category
    \end{itemize}
  \item
    education: (categorical) Bachelors, Some-college, 11th, HS-grad,
    Prof-school, Assoc-acdm, Assoc-voc, 9th, 7th-8th, 12th, Masters,
    1st-4th, 10th, Doctorate, 5th-6th, Preschool.
  \item
    Individual's highest education degree
  \item
    marital-status: (categorical) Married-civ-spouse, Divorced,
    Never-married, Separated, Widowed, Married-spouse-absent,
    Married-AF-spouse.

    \begin{itemize}
    \tightlist
    \item
      Individual marital status
    \end{itemize}
  \item
    occupation: (categorical) Tech-support, Craft-repair, Other-service,
    Sales, Exec-managerial, Prof-specialty, Handlers-cleaners,
    Machine-op-inspct, Adm-clerical, Farming-fishing, Transport-moving,
    Priv-house-serv, Protective-serv, Armed-Forces.

    \begin{itemize}
    \tightlist
    \item
      Individual's occupation
    \end{itemize}
  \item
    relationship: (categorical) Wife, Own-child, Husband, Not-in-family,
    Other-relative, Unmarried.

    \begin{itemize}
    \tightlist
    \item
      Individual's relation in a family
    \end{itemize}
  \item
    race: (categorical) White, Asian-Pac-Islander, Amer-Indian-Eskimo,
    Other, Black.

    \begin{itemize}
    \tightlist
    \item
      Race of Individual
    \end{itemize}
  \item
    sex: (categorical) Female, Male.
  \item
    native-country: (categorical) United-States, Cambodia, England,
    Puerto-Rico, Canada, Germany, Outlying-US(Guam-USVI-etc), India,
    Japan, Greece, South, China, Cuba, Iran, Honduras, Philippines,
    Italy, Poland, Jamaica, Vietnam, Mexico, Portugal, Ireland, France,
    Dominican-Republic, Laos, Ecuador, Taiwan, Haiti, Columbia, Hungary,
    Guatemala, Nicaragua, Scotland, Thailand, Yugoslavia, El-Salvador,
    Trinadad\&Tobago, Peru, Hong, Holand-Netherlands.

    \begin{itemize}
    \tightlist
    \item
      Individual's native country
    \end{itemize}
  \end{itemize}
\item
  Continuous Attributes

  \begin{itemize}
  \tightlist
  \item
    age: continuous.

    \begin{itemize}
    \tightlist
    \item
      Age of an individual
    \end{itemize}
  \item
    education-num: number of education year, continuous.

    \begin{itemize}
    \tightlist
    \item
      Individual's year of receiving education
    \end{itemize}
  \item
    fnlwgt: final weight, continuous.

    \begin{itemize}
    \tightlist
    \item
      The weights on the CPS files are controlled to independent
      estimates of the civilian noninstitutional population of the US.
      These are prepared monthly for us by Population Division here at
      the Census Bureau.
    \end{itemize}
  \item
    capital-gain: continuous.
  \item
    capital-loss: continuous.
  \item
    hours-per-week: continuous.

    \begin{itemize}
    \tightlist
    \item
      Individual's working hour per week
    \end{itemize}
  \end{itemize}
\end{enumerate}

Check if there are any NaNs in the dataframe and count every columns

    \begin{Verbatim}[commandchars=\\\{\}]
{\color{incolor}In [{\color{incolor}4}]:} \PY{n}{original\PYZus{}data}\PY{o}{.}\PY{n}{isnull}\PY{p}{(}\PY{p}{)}\PY{o}{.}\PY{n}{sum}\PY{p}{(}\PY{p}{)}
\end{Verbatim}


\begin{Verbatim}[commandchars=\\\{\}]
{\color{outcolor}Out[{\color{outcolor}4}]:} Age                  0
        Workclass         1836
        fnlwgt               0
        Education            0
        Education-Num        0
        Martial Status       0
        Occupation        1843
        Relationship         0
        Race                 0
        Sex                  0
        Capital Gain         0
        Capital Loss         0
        Hours per week       0
        Country            583
        Target               0
        dtype: int64
\end{Verbatim}
            
\subsection{Histogram Analysis}\label{histogram-analysis}

A histogram is an accurate representation of the distribution of
numerical data. It is an estimate of the probability distribution of a
continuous variable (quantitative variable) and was first introduced by
Karl Pearson. It differs from a bar graph, in the sense that a bar graph
relates two variables, but a histogram relates only one. To construct a
histogram, the first step is to "bin" (or "bucket") the range of
values---that is, divide the entire range of values into a series of
intervals---and then count how many values fall into each interval. The
bins are usually specified as consecutive, non-overlapping intervals of
a variable. The bins (intervals) must be adjacent, and are often (but
are not required to be) of equal size.

Histogram can be summarized roughly as an inventory of what "kinds of
items" you have and "how many of each kind" you have. In computer
vision, histogram appears a lot and many times helps to introduce some
sort of robustness to your method. For example a bunch of techniques
called local features/descriptors make use of the histogram of the image
gradient in an image region. This summary representation helps you
compare different images without being affected too much by variations
in pixel values, shifts and tilts, etc. that change the individual pixel
values significantly. So, histogram has the benefit of a summary data
structure that is robust to certain changes that you want to ignore in
the raw data.

    \begin{Verbatim}[commandchars=\\\{\}]
{\color{incolor}In [{\color{incolor}5}]:} \PY{k}{def} \PY{n+nf}{make\PYZus{}histogram}\PY{p}{(}\PY{n}{df}\PY{p}{)}\PY{p}{:}
            \PY{n}{fig} \PY{o}{=} \PY{n}{plt}\PY{o}{.}\PY{n}{figure}\PY{p}{(}\PY{n}{figsize}\PY{o}{=}\PY{p}{(}\PY{l+m+mi}{20}\PY{p}{,}\PY{l+m+mi}{35}\PY{p}{)}\PY{p}{)}
            \PY{n}{COL} \PY{o}{=} \PY{l+m+mi}{3}
            \PY{n}{ROW} \PY{o}{=} \PY{n}{math}\PY{o}{.}\PY{n}{ceil}\PY{p}{(}\PY{n+nb}{float}\PY{p}{(}\PY{n}{df}\PY{o}{.}\PY{n}{shape}\PY{p}{[}\PY{l+m+mi}{1}\PY{p}{]}\PY{p}{)}\PY{o}{/}\PY{n}{COL}\PY{p}{)}
            
            \PY{k}{for} \PY{n}{i} \PY{p}{,} \PY{n}{column} \PY{o+ow}{in} \PY{n+nb}{enumerate}\PY{p}{(}\PY{n}{df}\PY{o}{.}\PY{n}{columns}\PY{p}{)}\PY{p}{:}
                \PY{n}{ax} \PY{o}{=} \PY{n}{fig}\PY{o}{.}\PY{n}{add\PYZus{}subplot}\PY{p}{(}\PY{n}{ROW}\PY{p}{,} \PY{n}{COL}\PY{p}{,} \PY{n}{i}\PY{o}{+}\PY{l+m+mi}{1}\PY{p}{)}
                \PY{n}{ax}\PY{o}{.}\PY{n}{set\PYZus{}title}\PY{p}{(}\PY{n}{column}\PY{p}{)}
                \PY{k}{if} \PY{n}{df}\PY{o}{.}\PY{n}{dtypes}\PY{p}{[}\PY{n}{column}\PY{p}{]} \PY{o}{==} \PY{n}{np}\PY{o}{.}\PY{n}{object}\PY{p}{:}
                    \PY{n}{df}\PY{p}{[}\PY{n}{column}\PY{p}{]}\PY{o}{.}\PY{n}{value\PYZus{}counts}\PY{p}{(}\PY{p}{)}\PY{o}{.}\PY{n}{plot}\PY{p}{(}\PY{n}{kind}\PY{o}{=}\PY{l+s+s2}{\PYZdq{}}\PY{l+s+s2}{bar}\PY{l+s+s2}{\PYZdq{}}\PY{p}{,} \PY{n}{axes} \PY{o}{=} \PY{n}{ax}\PY{p}{)}
                \PY{k}{else}\PY{p}{:}
                    \PY{n}{df}\PY{p}{[}\PY{n}{column}\PY{p}{]}\PY{o}{.}\PY{n}{hist}\PY{p}{(}\PY{n}{axes} \PY{o}{=} \PY{n}{ax}\PY{p}{)}
                    \PY{n}{plt}\PY{o}{.}\PY{n}{xticks}\PY{p}{(}\PY{n}{rotation}\PY{o}{=}\PY{l+s+s2}{\PYZdq{}}\PY{l+s+s2}{vertical}\PY{l+s+s2}{\PYZdq{}}\PY{p}{)}
                    
            \PY{n}{plt}\PY{o}{.}\PY{n}{subplots\PYZus{}adjust}\PY{p}{(}\PY{n}{hspace}\PY{o}{=}\PY{l+m+mf}{0.7}\PY{p}{,} \PY{n}{wspace}\PY{o}{=}\PY{l+m+mf}{0.2}\PY{p}{)}
            
        \PY{n}{make\PYZus{}histogram}\PY{p}{(}\PY{n}{original\PYZus{}data}\PY{p}{)}
\end{Verbatim}


    \begin{center}
    \adjustimage{max size={0.9\linewidth}{0.9\paperheight}}{tugas-1_files/tugas-1_12_0.png}
    \end{center}
    { \hspace*{\fill} \\}
    
The histograms above shows that all of the data do not have a normal
distribution, therefore requiring special methods to deal with the
missing value.

The \texttt{Country} feature analysis is described below.

    \begin{Verbatim}[commandchars=\\\{\}]
{\color{incolor}In [{\color{incolor}6}]:} \PY{p}{(}\PY{n}{original\PYZus{}data}\PY{p}{[}\PY{l+s+s2}{\PYZdq{}}\PY{l+s+s2}{Country}\PY{l+s+s2}{\PYZdq{}}\PY{p}{]}\PY{o}{.}\PY{n}{value\PYZus{}counts}\PY{p}{(}\PY{p}{)} \PY{o}{/} \PY{n}{original\PYZus{}data}\PY{o}{.}\PY{n}{shape}\PY{p}{[}\PY{l+m+mi}{0}\PY{p}{]}\PY{p}{)}\PY{o}{.}\PY{n}{head}\PY{p}{(}\PY{p}{)}
\end{Verbatim}


\begin{Verbatim}[commandchars=\\\{\}]
{\color{outcolor}Out[{\color{outcolor}6}]:} United-States    0.895857
        Mexico           0.019748
        Philippines      0.006081
        Germany          0.004207
        Canada           0.003716
        Name: Country, dtype: float64
\end{Verbatim}
            
Indeed! 89\% of the samples are for people from the US. Mexico comes
next with less than 2\%.

\subsection{Boxplot Analysis}\label{boxplot-analysis}

Boxplot is a method for graphically depicting groups of numerical data
through their quartiles. Box plots handle large data effortlessly, but
they do not retain the exact values and the details of the results of
the distribution. These graphs allow a clear summary of large amounts of
data.

    \begin{Verbatim}[commandchars=\\\{\}]
{\color{incolor}In [{\color{incolor}7}]:} \PY{k}{def} \PY{n+nf}{make\PYZus{}boxplot}\PY{p}{(}\PY{n}{df}\PY{p}{)}\PY{p}{:}
            \PY{n}{fig} \PY{o}{=} \PY{n}{plt}\PY{o}{.}\PY{n}{figure}\PY{p}{(}\PY{n}{figsize}\PY{o}{=}\PY{p}{(}\PY{l+m+mi}{20}\PY{p}{,}\PY{l+m+mi}{35}\PY{p}{)}\PY{p}{)}
            \PY{n}{COL} \PY{o}{=} \PY{l+m+mi}{3}
            \PY{n}{ROW} \PY{o}{=} \PY{n}{math}\PY{o}{.}\PY{n}{ceil}\PY{p}{(}\PY{n+nb}{float}\PY{p}{(}\PY{n}{df}\PY{o}{.}\PY{n}{shape}\PY{p}{[}\PY{l+m+mi}{1}\PY{p}{]}\PY{p}{)}\PY{o}{/}\PY{n}{COL}\PY{p}{)}
            
            \PY{n}{iterator} \PY{o}{=} \PY{l+m+mi}{1}
            \PY{k}{for} \PY{n}{column} \PY{o+ow}{in} \PY{n}{df}\PY{o}{.}\PY{n}{columns}\PY{p}{:}
                \PY{k}{if} \PY{n}{df}\PY{o}{.}\PY{n}{dtypes}\PY{p}{[}\PY{n}{column}\PY{p}{]} \PY{o}{!=} \PY{n}{np}\PY{o}{.}\PY{n}{object}\PY{p}{:}
                    \PY{n}{ax} \PY{o}{=} \PY{n}{fig}\PY{o}{.}\PY{n}{add\PYZus{}subplot}\PY{p}{(}\PY{n}{ROW}\PY{p}{,} \PY{n}{COL}\PY{p}{,} \PY{n}{iterator}\PY{p}{)}
                    \PY{n}{ax}\PY{o}{.}\PY{n}{set\PYZus{}title}\PY{p}{(}\PY{n}{column}\PY{p}{)}
                    \PY{n}{pd}\PY{o}{.}\PY{n}{DataFrame}\PY{p}{(}\PY{n}{df}\PY{p}{[}\PY{n}{column}\PY{p}{]}\PY{p}{,} \PY{n}{columns}\PY{o}{=}\PY{p}{[}\PY{n}{column}\PY{p}{]}\PY{p}{)}\PY{o}{.}\PY{n}{boxplot}\PY{p}{(}\PY{p}{)}
                    \PY{n}{iterator}\PY{o}{+}\PY{o}{=}\PY{l+m+mi}{1}
                    
            \PY{n}{plt}\PY{o}{.}\PY{n}{subplots\PYZus{}adjust}\PY{p}{(}\PY{n}{hspace}\PY{o}{=}\PY{l+m+mf}{0.7}\PY{p}{,} \PY{n}{wspace}\PY{o}{=}\PY{l+m+mf}{0.2}\PY{p}{)}
            \PY{n}{plt}\PY{o}{.}\PY{n}{show}\PY{p}{(}\PY{p}{)}
        
        \PY{n}{make\PYZus{}boxplot}\PY{p}{(}\PY{n}{original\PYZus{}data}\PY{p}{)}
\end{Verbatim}


    \begin{center}
    \adjustimage{max size={0.9\linewidth}{0.9\paperheight}}{tugas-1_files/tugas-1_17_0.png}
    \end{center}
    { \hspace*{\fill} \\}
    
The Boxplot shows that some of the data have many outlier values. This
is still acceptable as the main data because these data are consisted of
categorical data types.

\subsection{Correlation Analysis}\label{correlation-analysis}

We also need to do data correlation analysis to figure out the
correlation between each feature inside the dataset. Below are the
\texttt{pairplot} analysis of each features in the dataset.

    \begin{Verbatim}[commandchars=\\\{\}]
{\color{incolor}In [{\color{incolor}8}]:} \PY{n}{sns}\PY{o}{.}\PY{n}{set}\PY{p}{(}\PY{n}{style}\PY{o}{=}\PY{l+s+s2}{\PYZdq{}}\PY{l+s+s2}{ticks}\PY{l+s+s2}{\PYZdq{}}\PY{p}{)}
        \PY{n}{sns}\PY{o}{.}\PY{n}{pairplot}\PY{p}{(}\PY{n}{original\PYZus{}data}\PY{p}{,} \PY{n}{hue}\PY{o}{=}\PY{l+s+s1}{\PYZsq{}}\PY{l+s+s1}{Target}\PY{l+s+s1}{\PYZsq{}}\PY{p}{)}
        \PY{n}{plt}\PY{o}{.}\PY{n}{show}\PY{p}{(}\PY{p}{)}
\end{Verbatim}


    \begin{center}
    \adjustimage{max size={0.9\linewidth}{0.9\paperheight}}{tugas-1_files/tugas-1_20_0.png}
    \end{center}
    { \hspace*{\fill} \\}
    
Below are the data correlation analysis using \texttt{heatmap} analysis.

    \begin{Verbatim}[commandchars=\\\{\}]
{\color{incolor}In [{\color{incolor}9}]:} \PY{c+c1}{\PYZsh{} Encode the categorical features as numbers}
        \PY{k}{def} \PY{n+nf}{number\PYZus{}encode\PYZus{}features}\PY{p}{(}\PY{n}{df}\PY{p}{)}\PY{p}{:}
            \PY{n}{result} \PY{o}{=} \PY{n}{df}\PY{o}{.}\PY{n}{copy}\PY{p}{(}\PY{p}{)}
            \PY{n}{encoders} \PY{o}{=} \PY{p}{\PYZob{}}\PY{p}{\PYZcb{}}
            \PY{k}{for} \PY{n}{column} \PY{o+ow}{in} \PY{n}{result}\PY{o}{.}\PY{n}{columns}\PY{p}{:}
                \PY{k}{if} \PY{n}{result}\PY{o}{.}\PY{n}{dtypes}\PY{p}{[}\PY{n}{column}\PY{p}{]} \PY{o}{==} \PY{n}{np}\PY{o}{.}\PY{n}{object}\PY{p}{:}
                    \PY{n}{encoders}\PY{p}{[}\PY{n}{column}\PY{p}{]} \PY{o}{=} \PY{n}{preprocessing}\PY{o}{.}\PY{n}{LabelEncoder}\PY{p}{(}\PY{p}{)}
                    \PY{n}{result}\PY{p}{[}\PY{n}{column}\PY{p}{]} \PY{o}{=} \PY{n}{encoders}\PY{p}{[}\PY{n}{column}\PY{p}{]}\PY{o}{.}\PY{n}{fit\PYZus{}transform}\PY{p}{(}\PY{n}{result}\PY{p}{[}\PY{n}{column}\PY{p}{]}\PY{o}{.}\PY{n}{fillna}\PY{p}{(}\PY{l+s+s1}{\PYZsq{}}\PY{l+s+s1}{0}\PY{l+s+s1}{\PYZsq{}}\PY{p}{)}\PY{p}{)}
            \PY{k}{return} \PY{n}{result}\PY{p}{,} \PY{n}{encoders}
        
        \PY{c+c1}{\PYZsh{} Calculate the correlation and plot it}
        \PY{n}{encoded\PYZus{}data}\PY{p}{,} \PY{n}{\PYZus{}} \PY{o}{=} \PY{n}{number\PYZus{}encode\PYZus{}features}\PY{p}{(}\PY{n}{original\PYZus{}data}\PY{p}{)}
        \PY{n}{sns}\PY{o}{.}\PY{n}{heatmap}\PY{p}{(}\PY{n}{encoded\PYZus{}data}\PY{o}{.}\PY{n}{corr}\PY{p}{(}\PY{p}{)}\PY{p}{,} \PY{n}{square}\PY{o}{=}\PY{k+kc}{True}\PY{p}{)}
        \PY{n}{plt}\PY{o}{.}\PY{n}{show}\PY{p}{(}\PY{p}{)}
\end{Verbatim}


    \begin{center}
    \adjustimage{max size={0.9\linewidth}{0.9\paperheight}}{tugas-1_files/tugas-1_22_0.png}
    \end{center}
    { \hspace*{\fill} \\}
    
The heatplot above shows that there is a high correlation between
\texttt{Education} and \texttt{Education-Num}.

    \begin{Verbatim}[commandchars=\\\{\}]
{\color{incolor}In [{\color{incolor}10}]:} \PY{n}{original\PYZus{}data}\PY{p}{[}\PY{p}{[}\PY{l+s+s2}{\PYZdq{}}\PY{l+s+s2}{Education}\PY{l+s+s2}{\PYZdq{}}\PY{p}{,} \PY{l+s+s2}{\PYZdq{}}\PY{l+s+s2}{Education\PYZhy{}Num}\PY{l+s+s2}{\PYZdq{}}\PY{p}{]}\PY{p}{]}\PY{o}{.}\PY{n}{head}\PY{p}{(}\PY{l+m+mi}{15}\PY{p}{)}
\end{Verbatim}


\begin{Verbatim}[commandchars=\\\{\}]
{\color{outcolor}Out[{\color{outcolor}10}]:}        Education  Education-Num
         0      Bachelors             13
         1      Bachelors             13
         2        HS-grad              9
         3           11th              7
         4      Bachelors             13
         5        Masters             14
         6            9th              5
         7        HS-grad              9
         8        Masters             14
         9      Bachelors             13
         10  Some-college             10
         11     Bachelors             13
         12     Bachelors             13
         13    Assoc-acdm             12
         14     Assoc-voc             11
\end{Verbatim}
            
Two columns \texttt{Education} and \texttt{Education-Num} actually
represent the same features, but encoded as strings and as numbers. We
don't need the string representation, so we can just delete this column.
Note that it is a much better option to delete the Education column as
the Education-Num has the important property that the values are
ordered: the higher the number, the higher the education that person
has. This is a vaulable information a machine learning algorithm can
use.

\section{Data Preprocessing}\label{data-preprocessing}

The preprocessing that will be carried out Imputation using
\texttt{Simpleimputer}. To replace the missing values in the categorical
data, we will use the mode or the most frequent value that appeared in
each column. On the \texttt{SimpleInputer} method, this is carried out
using the
\texttt{strategy=\textquotesingle{}most\_frequent\textquotesingle{}} as
the parameter.

    \begin{Verbatim}[commandchars=\\\{\}]
{\color{incolor}In [{\color{incolor}11}]:} \PY{n}{imputer\PYZus{}modus} \PY{o}{=} \PY{n}{SimpleImputer}\PY{p}{(}\PY{n}{missing\PYZus{}values}\PY{o}{=}\PY{n}{np}\PY{o}{.}\PY{n}{nan}\PY{p}{,} \PY{n}{strategy}\PY{o}{=}\PY{l+s+s1}{\PYZsq{}}\PY{l+s+s1}{most\PYZus{}frequent}\PY{l+s+s1}{\PYZsq{}}\PY{p}{)}
         \PY{n}{imputer\PYZus{}modus}\PY{o}{.}\PY{n}{fit}\PY{p}{(}\PY{n}{original\PYZus{}data}\PY{p}{)}
         \PY{n}{imputed\PYZus{}data} \PY{o}{=} \PY{n}{imputer\PYZus{}modus}\PY{o}{.}\PY{n}{transform}\PY{p}{(}\PY{n}{original\PYZus{}data}\PY{p}{)}
         
         \PY{n}{imputed\PYZus{}dataframe} \PY{o}{=} \PY{n}{pd}\PY{o}{.}\PY{n}{DataFrame}\PY{p}{(}\PY{n}{imputed\PYZus{}data}\PY{p}{,}
             \PY{n}{columns}\PY{o}{=}\PY{p}{[}\PY{l+s+s2}{\PYZdq{}}\PY{l+s+s2}{Age}\PY{l+s+s2}{\PYZdq{}}\PY{p}{,} \PY{l+s+s2}{\PYZdq{}}\PY{l+s+s2}{Workclass}\PY{l+s+s2}{\PYZdq{}}\PY{p}{,} \PY{l+s+s2}{\PYZdq{}}\PY{l+s+s2}{fnlwgt}\PY{l+s+s2}{\PYZdq{}}\PY{p}{,} \PY{l+s+s2}{\PYZdq{}}\PY{l+s+s2}{Education}\PY{l+s+s2}{\PYZdq{}}\PY{p}{,} \PY{l+s+s2}{\PYZdq{}}\PY{l+s+s2}{Education\PYZhy{}Num}\PY{l+s+s2}{\PYZdq{}}\PY{p}{,} \PY{l+s+s2}{\PYZdq{}}\PY{l+s+s2}{Martial Status}\PY{l+s+s2}{\PYZdq{}}\PY{p}{,}
                 \PY{l+s+s2}{\PYZdq{}}\PY{l+s+s2}{Occupation}\PY{l+s+s2}{\PYZdq{}}\PY{p}{,} \PY{l+s+s2}{\PYZdq{}}\PY{l+s+s2}{Relationship}\PY{l+s+s2}{\PYZdq{}}\PY{p}{,} \PY{l+s+s2}{\PYZdq{}}\PY{l+s+s2}{Race}\PY{l+s+s2}{\PYZdq{}}\PY{p}{,} \PY{l+s+s2}{\PYZdq{}}\PY{l+s+s2}{Sex}\PY{l+s+s2}{\PYZdq{}}\PY{p}{,} \PY{l+s+s2}{\PYZdq{}}\PY{l+s+s2}{Capital Gain}\PY{l+s+s2}{\PYZdq{}}\PY{p}{,} \PY{l+s+s2}{\PYZdq{}}\PY{l+s+s2}{Capital Loss}\PY{l+s+s2}{\PYZdq{}}\PY{p}{,}
                 \PY{l+s+s2}{\PYZdq{}}\PY{l+s+s2}{Hours per week}\PY{l+s+s2}{\PYZdq{}}\PY{p}{,} \PY{l+s+s2}{\PYZdq{}}\PY{l+s+s2}{Country}\PY{l+s+s2}{\PYZdq{}}\PY{p}{,} \PY{l+s+s2}{\PYZdq{}}\PY{l+s+s2}{Target}\PY{l+s+s2}{\PYZdq{}}\PY{p}{]}\PY{p}{)}
         \PY{n}{imputed\PYZus{}dataframe}\PY{o}{.}\PY{n}{head}\PY{p}{(}\PY{p}{)}
\end{Verbatim}


\begin{Verbatim}[commandchars=\\\{\}]
{\color{outcolor}Out[{\color{outcolor}11}]:}   Age         Workclass  fnlwgt  Education Education-Num      Martial Status  \textbackslash{}
         0  39         State-gov   77516  Bachelors            13       Never-married   
         1  50  Self-emp-not-inc   83311  Bachelors            13  Married-civ-spouse   
         2  38           Private  215646    HS-grad             9            Divorced   
         3  53           Private  234721       11th             7  Married-civ-spouse   
         4  28           Private  338409  Bachelors            13  Married-civ-spouse   
         
                   Occupation   Relationship   Race     Sex Capital Gain Capital Loss  \textbackslash{}
         0       Adm-clerical  Not-in-family  White    Male         2174            0   
         1    Exec-managerial        Husband  White    Male            0            0   
         2  Handlers-cleaners  Not-in-family  White    Male            0            0   
         3  Handlers-cleaners        Husband  Black    Male            0            0   
         4     Prof-specialty           Wife  Black  Female            0            0   
         
           Hours per week        Country Target  
         0             40  United-States  <=50K  
         1             13  United-States  <=50K  
         2             40  United-States  <=50K  
         3             40  United-States  <=50K  
         4             40           Cuba  <=50K  
\end{Verbatim}
            

    % Add a bibliography block to the postdoc
    
    
    
    \end{document}
